\documentclass[12pt,a4paper,titlepage]{article}
\label{Packages}
\RequirePackage{amssymb}
\RequirePackage{amsthm}
\RequirePackage[T1]{fontenc} 							%
\RequirePackage[utf8]{inputenc} 						% Umlaute
\RequirePackage[ngerman]{babel}							% Rechtschreibprüfung 
\RequirePackage{graphicx}								% Package um Bilder zu laden 
\RequirePackage{setspace}								%
\RequirePackage{fancyhdr}								% Paket für Kopfzeilen 
\RequirePackage{titlesec}								%
\RequirePackage{tocloft}								%
\RequirePackage[a4paper]{geometry}						% Seitenränder
\usepackage[linesnumbered,ruled]{algorithm2e}
\RequirePackage{array}
\RequirePackage[bookmarksopen=true]{hyperref}
\RequirePackage{bookmark}
\usepackage{eurosym}
\usepackage{pdfpages}
\usepackage{tcolorbox}
\tcbuselibrary{theorems}
\usepackage{graphicx}
\usepackage{caption}
\usepackage{subcaption}
\usepackage[procnames]{listings}
\usepackage{color}

%\newtcbtheorem[number within=section]{theo}{Theorem}%
%{colback=red!5,colframe=green!35!black,fonttitle=\bfseries}{th}
%\newtcbtheorem[number within=section]{defi}{Definition}%
%{colback=red!5,colframe=red!35!black,fonttitle=\bfseries}{th}

\usepackage{listings}


\newcounter{nalg} % defines algorithm counter for chapter-level
\renewcommand{\thenalg}{\thechapter \arabic{nalg}} %defines appearance of the algorithm counter
\DeclareCaptionLabelFormat{algocaption}{Algorithmus \thenalg} % defines a new caption label as Algorithm x.y

\lstnewenvironment{algo}[1][] %defines the algorithm listing environment
{   
    \refstepcounter{nalg} %increments algorithm number
    \captionsetup{labelformat=algocaption,labelsep=colon} %defines the caption setup for: it ises label format as the declared caption label above and makes label and caption text to be separated by a ':'
    \lstset{ %this is the stype
        frame=tB,
        numbers=left, 
        numberstyle=\tiny,
        basicstyle=\footnotesize, 
        keywordstyle=\color{black}\bfseries\em,
        keywords={,input, output, return, datatype, function, in, if, else, foreach, while, begin, end, } %add the keywords you want, or load a language as Rubens explains in his comment above.
        numbers=left,
        xleftmargin=.04\textwidth,
        #1 % this is to add specific settings to an usage of this environment (for instnce, the caption and referable label)
    }
}
{}




%%%%%%%%%%%%%%%%%%%%%%%%%%%%%%%%%%%%%%%%%%%%%%%%%%%%%%%%%%%%%%%%%%%%%%%%%%%%%%%%%%%%%%%%%%
%%%%%%%%%%%%%%%%%%%%%%%%%%%%%%%%%%%   PageFormat   %%%%%%%%%%%%%%%%%%%%%%%%%%%%%%%%%%%%%%%
%%%%%%%%%%%%%%%%%%%%%%%%%%%%%%%%%%%%%%%%%%%%%%%%%%%%%%%%%%%%%%%%%%%%%%%%%%%%%%%%%%%%%%%%%%

%%% Seitenränder anpassen 
\setlength{\topskip}{\ht\strutbox} 					% behebt Warnung von geometry 
\setlength{\headheight}{1.1\baselineskip}   		% Headheight hochsetzten 
\geometry{paper=a4paper,
left=25mm,											% Seitenabstand Links
right=15mm,											% Seitenabstand Rechts
top=25mm,											% Seitenabstand Oben
bottom=25mm} 										% Seitenabstand Unten

%%%%%%%%%%%%%%%%%%%%%%%%%%%%%%%%%%%%%%%%%%%%%%%%%%%%%%%%%%%%%%%%%%%%%%%%%%%%%%%%%%%%%%%%%%
%%%%%%%%%%%%%%%%%%%%%%%%%%%%%%%%%%   Pagestyles  %%%%%%%%%%%%%%%%%%%%%%%%%%%%%%%%%%%%%%%%%
%%%%%%%%%%%%%%%%%%%%%%%%%%%%%%%%%%%%%%%%%%%%%%%%%%%%%%%%%%%%%%%%%%%%%%%%%%%%%%%%%%%%%%%%%%

\label{Commands/style}

\newtheorem{definition}{Definition}[section]
\newtheorem{theorem}{Theorem}[section]
\newtheorem{corollary}{Corollary}[theorem]
\newtheorem{lemma}[theorem]{Lemma}

\newcommand*{\Union}{\bigcup}
\newcommand{\cs}[1]{\textit{\mathcal{C}(\mathcal{R})}}
\newcommand{\ws}[1]{\textit{Work Space}}
\newcommand{\fors}[1]{\textit{Forbidden Space}}
\newcommand{\frees}[1]{\textit{Free Space}}


%%% Fancy
\pagestyle{fancy} 
\renewcommand{\headrulewidth}{0pt}
\lfoot{}
\cfoot{}
\rfoot{\thepage}

\begin{document}

\begin{titlepage}
\label{01: titlepage}

\newcommand{\HRule}{\rule{\linewidth}{0.5mm}} % Defines a new command for the horizontal lines, change thickness here

\center % Center everything on the page
 
%----------------------------------------------------------------------------------------
%	HEADING SECTIONS
%----------------------------------------------------------------------------------------
\label{Heading}
\includegraphics[width=8cm]{pictures/logo.png}\\[1.5cm] % Name of your university/college
\textsc{\Large Bachelorarbeit}\\[0.5cm] % Major heading such as course name

%----------------------------------------------------------------------------------------
%	TITLE SECTION
%----------------------------------------------------------------------------------------
\label{Title Section}
\HRule \\[0.4cm]
{ \huge \bfseries Robot Motion Planning}\\[0.4cm] % Title of your document
\HRule \\[1.5cm]
 
%----------------------------------------------------------------------------------------
%	AUTHOR SECTION
%----------------------------------------------------------------------------------------
\label{Author}
\begin{minipage}{0.4\textwidth}
\begin{flushleft} \large
\emph{Author:}\\
Julian \textsc{Lorenz}\\
Mtr.: 3383863 % Your name
\end{flushleft}
\end{minipage}
~
\begin{minipage}{0.4\textwidth}
\begin{flushright} \large
\emph{Supervisor:} \\
Dr. Annamaria \textsc{Kovacs}\\
David \textsc{Veith}
% Supervisor's Name
\end{flushright}
\end{minipage}\\[4cm]


%----------------------------------------------------------------------------------------
%	DATE SECTION
%----------------------------------------------------------------------------------------
\label{Date}
{\large \today}\\[3cm] % Date, change the \today to a set date if you want to be precise

%----------------------------------------------------------------------------------------
%	LOGO SECTION
%----------------------------------------------------------------------------------------
\label{Logo}
%\includegraphics{numb5.png}\\[1cm] % Include a department/university logo - this will require the graphicx package
 
%----------------------------------------------------------------------------------------

\vfill % Fill the rest of the page with whitespace
\end{titlepage}

%%%%%%%%%%%%%%%%%%%%%%%%%%%%%%%%%%%%%%%%%%%%%%%%%%%%%%%%%%%%%%%%%%%%%%%%%%%%%%%%%%%%%%%%%%
\section{Einleitung}
\label{01: Einleitung}
Diese Seminararbeit behandelt das \textit{Robot Motion Planning} Problem aus  Kapitel 13 aus dem Buch \textit{Computational Geometry} \cite{bla}.
Es beschäftigt sich mit der abstrakten Planung der Bewegungen eines Roboters in einem vorerst abgeschlossenen Raum. Innerhalb des Raumes existieren statische Hindernisse, mit denen der Roboter nicht kollidieren darf.\\[.2cm]
Der Einfachheit halbe




%%%%%%%%%%%%%%%%%%%%%%%%%%%%%%%%%%%%%%%%%%%%%%%%%%%%%%%%%%%%%%%%%%%%%%%%%%%%%%%%%%%%%%%%%%
%								algo test
%%%%%%%%%%%%%%%%%%%%%%%%%%%%%%%%%%%%%%%%%%%%%%%%%%%%%%%%%%%%%%%%%%%%%%%%%%%%%%%%%%%%%%%%%%
\label{alg: Test}
\begin{algorithm}
    \SetKwInOut{Input}{Input}
    \SetKwInOut{Output}{Output}

    \underline{function Euclid} $(a,b)$\;
    \Input{Two nonnegative integers $a$ and $b$}
    \Output{$\gcd(a,b)$}
    \eIf{$b=0$}
      {
        return $a$\;
      }
      {
        return Euclid$(b,a\mod b)$\;
      }
    \caption{Euclid's algorithm for finding the greatest common divisor of two nonnegative integers}
\end{algorithm}

\input{bla.tex}




%%%%%%%%%%%%%%%%%%%%%%%%%%%%%%%%%%%%%%%%%%%%%%%%%%%%%%%%%%%%%%%%%%%%%%%%%%%%%%%%%%%%%%%%%%
%								Biblopgraphy
%%%%%%%%%%%%%%%%%%%%%%%%%%%%%%%%%%%%%%%%%%%%%%%%%%%%%%%%%%%%%%%%%%%%%%%%%%%%%%%%%%%%%%%%%%
\label{Biography}
\begin{thebibliography}{9}
\bibitem{latexcompanion} 
Michel Goossens, Frank Mittelbach, and Alexander Samarin. 
\textit{The \LaTeX\ Companion}. 
Addison-Wesley, Reading, Massachusetts, 1993.
 
\bibitem{einstein} 
Albert Einstein. 
\textit{Zur Elektrodynamik bewegter K{\"o}rper}. (German) 
[\textit{On the electrodynamics of moving bodies}]. 
Annalen der Physik, 322(10):891–921, 1905.
 
\bibitem{knuthwebsite} 
Knuth: Computers and Typesetting,
\\\texttt{http://www-cs-faculty.stanford.edu/\~{}uno/abcde.html}
\end{thebibliography}
\end{document}