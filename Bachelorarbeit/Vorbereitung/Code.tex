\documentclass[12pt,a4paper,titlepage,xcolor=dvipsnames]{article}
%%%%%%%%%%%%%%%%%%%%%%%%%%%%%%%%%%%%%%%%%%%%%%%%%%%%%%%%%%%%%%%%%%%%%%%%%%%%%%%%%%%%%%%%%%%%%%%%%%%%%
\label{Packages}

\RequirePackage{amssymb}
\RequirePackage{amsthm}
\RequirePackage[T1]{fontenc} 							%
\RequirePackage[utf8]{inputenc} 						% Umlaute
\RequirePackage[ngerman]{babel}							% Rechtschreibprüfung 
\RequirePackage{graphicx}								% Package um Bilder zu laden 
\RequirePackage{setspace}								%
\RequirePackage{fancyhdr}								% Paket für Kopfzeilen 
\RequirePackage{titlesec}								%
\RequirePackage{tocloft}								%
\RequirePackage[a4paper]{geometry}						% Seitenränder
\usepackage[linesnumbered,ruled]{algorithm2e}
\RequirePackage{array}
\RequirePackage[bookmarksopen=true]{hyperref}
\RequirePackage{bookmark}
\usepackage{eurosym}
\usepackage{pdfpages}
\usepackage{tcolorbox}
\tcbuselibrary{theorems}
\usepackage{graphicx}
\usepackage{caption}
\usepackage{subcaption}
\usepackage[procnames]{listings}
\usepackage{color}
\usepackage{blindtext}
\usepackage{tabulary}
\usepackage{listings}
%%%%%%%%%%%%%%%%%%%%%%%%%%%%%%%%%%%%%%%%%%%%%%%%%%%%%%%%%%%%%%%%%%%%%%%%%%%%%%%%%%%%%%%%%%%%%%%%%%%%%
\label{Commands}

\newcounter{nalg} % defines algorithm counter for chapter-level
\renewcommand{\thenalg}{\thechapter \arabic{nalg}} %defines appearance of the algorithm counter
\DeclareCaptionLabelFormat{algocaption}{Algorithmus \thenalg} % defines a new caption label as Algorithm x.y

\lstnewenvironment{algo}[1][] %defines the algorithm listing environment
{   
    \refstepcounter{nalg} %increments algorithm number
    \captionsetup{labelformat=algocaption,labelsep=colon} %defines the caption setup for: it ises label format as the declared caption label above and makes label and caption text to be separated by a ':'
    \lstset{ %this is the stype
        frame=tB,
        numbers=left, 
        numberstyle=\tiny,
        basicstyle=\footnotesize, 
        keywordstyle=\color{black}\bfseries\em,
        keywords={,input, output, return, datatype, function, in, if, else, foreach, while, begin, end, } %add the keywords you want, or load a language as Rubens explains in his comment above.
        numbers=left,
        xleftmargin=.04\textwidth,
        #1 % this is to add specific settings to an usage of this environment (for instnce, the caption and referable label)
    }
}
{}




%%%%%%%%%%%%%%%%%%%%%%%%%%%%%%%%%%%%%%%%%%%%%%%%%%%%%%%%%%%%%%%%%%%%%%%%%%%%%%%%%%%%%%%%%%
%%%%%%%%%%%%%%%%%%%%%%%%%%%%%%%%%%%   PageFormat   %%%%%%%%%%%%%%%%%%%%%%%%%%%%%%%%%%%%%%%
%%%%%%%%%%%%%%%%%%%%%%%%%%%%%%%%%%%%%%%%%%%%%%%%%%%%%%%%%%%%%%%%%%%%%%%%%%%%%%%%%%%%%%%%%%
\label{Format}

%%% Seitenränder anpassen 
\setlength{\topskip}{\ht\strutbox} 					% behebt Warnung von geometry 
\setlength{\headheight}{1.1\baselineskip}   		% Headheight hochsetzten 
\geometry{paper=a4paper,
left=25mm,											% Seitenabstand Links
right=15mm,											% Seitenabstand Rechts
top=25mm,											% Seitenabstand Oben
bottom=25mm} 										% Seitenabstand Unten

%%%%%%%%%%%%%%%%%%%%%%%%%%%%%%%%%%%%%%%%%%%%%%%%%%%%%%%%%%%%%%%%%%%%%%%%%%%%%%%%%%%%%%%%%%
%%%%%%%%%%%%%%%%%%%%%%%%%%%%%%%%%%   Pagestyles  %%%%%%%%%%%%%%%%%%%%%%%%%%%%%%%%%%%%%%%%%
%%%%%%%%%%%%%%%%%%%%%%%%%%%%%%%%%%%%%%%%%%%%%%%%%%%%%%%%%%%%%%%%%%%%%%%%%%%%%%%%%%%%%%%%%%

\label{Commands/style}

\newtheorem{definition}{Definition}[section]
\newtheorem{theorem}{Theorem}[section]
\newtheorem{corollary}{Corollary}[theorem]
\newtheorem{lemma}[theorem]{Lemma}

\newcommand*{\Union}{\bigcup}
\newcommand{\cs}[1]{\textit{\mathcal{C}(\mathcal{R})}}
\newcommand{\ws}[1]{\textit{Work Space}}
\newcommand{\fors}[1]{\textit{Forbidden Space}}
\newcommand{\frees}[1]{\textit{Free Space}}

\definecolor{rahmen}{rgb}{.7,1,.7}

%%%%%%%%%%%%%%%%%%%%%%%%%%%%%%%%%%%%%%%%%%%%%%%%%%%%%%%%%%%%%%%%%%%%%%%%%%%%%%%%%%%%%%%%%%
%%%%%%%%%%%%%%%%%%%%%%%%%%%%%%%%%%   Header      %%%%%%%%%%%%%%%%%%%%%%%%%%%%%%%%%%%%%%%%%
%%%%%%%%%%%%%%%%%%%%%%%%%%%%%%%%%%%%%%%%%%%%%%%%%%%%%%%%%%%%%%%%%%%%%%%%%%%%%%%%%%%%%%%%%%
\label{Header}
%%% Fancy
\pagestyle{fancy} 
\renewcommand{\headrulewidth}{0pt}
\lhead{Julian Lorenz, 3383863\\Manuel Penschuck}
\chead{Effiziente Algorithmen\\Aufgabenblatt 2}
\rhead{\today\\ $\sum$ \rule[-2mm]{8mm}{0.2mm} / 23 Punkte}

\lfoot{}
\cfoot{}
\rfoot{\thepage}




%%%				Begin Document
\label{Begin}
\begin{document}
%%%%%%%%%%%%%%%%%%%%%%%%%%%%%%%%%%%%%%%%%%%%%%%%%%%%%%%%%%%%%%%%%%%%%%%%%%%%%%%%%%%%%%%%%%
\label{Aufgabe 1}
\noindent
\begin{tabulary}{17cm}{LLR}
\textbf{Aufgabe 2.1.} &\textsc{Paging}& ( \rule[-2mm]{5mm}{0.2mm} / 6 Punkte)\\
\hspace{2.8cm} &\hspace{10cm} &\hspace{4cm} 
\end{tabulary}
\noindent

$\Theta(n)$\\
$\mathcal{O}(n)$\\
$\mathcal{O}(n/B^{2/3}+sort(n)\cdot \log (B))$\\
$\Omega(n/B^{1/2}+sort(n))$\\
$G=(V,E)$\\
$scan(N)=\Theta(N/B)$\\
$sort(N)=\Theta((N/B)\cdot \log_{M/B}(N/B))$\\
$scan(N)<sort(N)\ll N$\\
$L(t):=A'(t)\backslash \{L(t-1)\cup L(t-2)\}$\\
$\mathcal{O}(n+m)$\\
$\Theta(n+m/B)$\\
$\mathcal{O}(\sum_t L(t)+\sum_t sort(A(t)))=\mathcal{O}(n+sort(n+m))$\\
$\mathcal{O}(sort(n+m))$\\
$\max\{1,\sqrt{\frac{n\cdot B}{n+m}}\}$\\
$1<\mu < \mathcal{O}(\sqrt{B})$\\
$\Omega(n/\mu)$\\
$\Omega(\mu)$\\
$\textbf{P}[r(v)=0]=\textbf{P}[r(v)=1]=\frac{1}{2}$\\
$\Delta d_i(v)=|d_{i-1}(v)-d_i(v)|$\\

\\
\vspace*{.2cm}
\noindent\rule{\textwidth}{1pt}
\\
%%%%%%%%%%%%%%%%%%%%%%%%%%%%%%%%%%%%%%%%%%%%%%%%%%%%%%%%%%%%%%%%%%%%%%%%%%%%%%%%%%%%%%%%%%
\label{Aufgabe 2}
\noindent
\begin{tabulary}{17cm}{LLR}
\textbf{Aufgabe 2.2.} &\textit{Erwartungswert}& ( \rule[-2mm]{5mm}{0.2mm} / 3 Punkte)\\
\hspace{2.8cm} &\hspace{10cm} &\hspace{4cm} 
\end{tabulary}
\noindent

\\
\vspace*{.2cm}
\noindent\rule{\textwidth}{1pt}
\\
%%%%%%%%%%%%%%%%%%%%%%%%%%%%%%%%%%%%%%%%%%%%%%%%%%%%%%%%%%%%%%%%%%%%%%%%%%%%%%%%%%%%%%%%%%
\label{Aufgabe 3}
\noindent
\begin{tabulary}{17cm}{LLR}
\textbf{Aufgabe 2.3.} &\textit{Kopf-oder-Zahl}& ( \rule[-2mm]{5mm}{0.2mm} / 6 Punkte)\\
\hspace{2.8cm} &\hspace{10cm} &\hspace{4cm} 
\end{tabulary}
\noindent


\\
\vspace*{.2cm}
\noindent\rule{\textwidth}{1pt}
\\
%%%%%%%%%%%%%%%%%%%%%%%%%%%%%%%%%%%%%%%%%%%%%%%%%%%%%%%%%%%%%%%%%%%%%%%%%%%%%%%%%%%%%%%%%%
\label{Aufgabe 4}
\noindent
\begin{tabulary}{17cm}{LLR}
\textbf{Aufgabe 1.4.} &\textsc{Unabhängige Ereignisse}& ( \rule[-2mm]{5mm}{0.2mm} / 8 Punkte)\\
\hspace{2.8cm} &\hspace{10cm} &\hspace{4cm} 
\end{tabulary}
\noindent


\label{End}
\end{document}
